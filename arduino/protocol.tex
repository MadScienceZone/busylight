% TODO: add support for Lumos-compatible RS-485 protocol
\input common
\frontmatter
\definecolor{reservedslot}{gray}{0.8}
\newcommand\api{\acronym{API}}
\newcommand\pc{\acronym{PC}}
\newcommand\cli{\acronym{CLI}}
\newcommand\ascii{\acronym{ASCII}}
\newcommand\led{\acronym{LED}}
\newcommand\codetype[1]{\z{#1}}
\newcommand\ixz[1]{\index{#1@\z{#1}}\z{#1}}
\newcommand\tUnused{\cellcolor{gray!50}}
\newcommand\tControl{\cellcolor{yellow!50}}
\newcommand\tForbidden{\cellcolor{red!50}}
\newcommand\tSpecial{\cellcolor{blue!25}}
%\colorlet{tableheader}{blue!40}
%\colorlet{tablesubhead}{blue!20}
%\colorlet{recordbox}{blue!15}
%\colorlet{recordtext}{black}
%\usetikzlibrary{positioning,shapes,shadows,arrows}
%\tikzstyle{program}=[rectangle, draw=black, rounded corners, fill=recordbox, drop shadow,
%	anchor=north, text=recordtext, text width=2cm]
%\tikzstyle{instance}=[rectangle, draw=black, rounded corners, fill=green!15, drop shadow,
%	anchor=north, text=black, text width=3cm]
%\tikzstyle{line}=[-, thick]
%\tikzstyle{myarrow}=[->, >=triangle 45, thick]
\hypersetup{
	pdftitle={Busylight User's Guide},
	pdfkeywords={Open Source Status Light Hardware and Software},
	pdfauthor={Steve Willoughby / Mad Science Zone},
	pdfsubject={Busylight Project Documentation * (c) 2023 * Creative Commons Licensing (See Document for details)},
	colorlinks=true,
	linkcolor=blue!30!black,
}
\thispagestyle{empty}
	\begin{center}
		\Huge Busylight \\ User's Guide \\
		WORKING\\DRAFT
	\end{center}

\vfill
%\end{flushright}
\newpage
The information in this document, and the hardware and software it describes, are hobbyist
works created as an educational exercise and as a matter of personal interest for recreational
purposes.

It is not to be considered an industrial-grade or retail-worthy product.
It is assumed that the user has the necessary understanding and skill to use it appropriately.  The author makes NO
representation as to suitability or fitness for any purpose whatsoever, and disclaims any and all liability or 
warranty to the full extent permitted by applicable law.  It is explicitly not designed for use where the safety
of persons, animals, property, or anything of real value depends on the correct operation of the software.

\strut\vfill

\begin{center}\bfseries
	For busylight hardware version 1.0.2, and firmware version 2.0.0.
\end{center}

\strut\vfill

\noindent Copyright \copyright\ 2023 by Steven L. Willoughby
(aka MadScienceZone), Aloha, Oregon, USA. All Rights Reserved.
This document is released under the terms and conditions of the
Creative Commons ``Attribution-NoDerivs 3.0 Unported'' license.
In summary, you are free to use, reproduce, and redistribute this 
document provided you give full attribution to its author and do not
alter it or create derivative works from it.  See
\begin{center}
\href{http://creativecommons.org/licenses/by-nd-3.0}{http://creativecommons.org/licenses/by\-nd\-/\-3.0/} 
\end{center}
for the full set of licensing terms.

\begin{center}
\LJimg[width=.25in]{cc}\LJimg[width=.25in]{by}\LJimg[width=.25in]{nd}
\end{center}

\newpage
\tableofcontents
\newpage
\listoffigures
\listoftables
\mainmatter

%%%%%%%%%%%%%%%%%%%%%%%%%%%%%%%%%%%%%%%%%%%%%%%%%%%%%%%%%%%%%%%%%%%%%%%%%%%%%%%%%%%%%%%%%%%%%%%%%%%%
%  ____  ____   ___ _____ ___   ____ ___  _     
% |  _ \|  _ \ / _ \_   _/ _ \ / ___/ _ \| |    
% | |_) | |_) | | | || || | | | |  | | | | |    
% |  __/|  _ <| |_| || || |_| | |__| |_| | |___ 
% |_|   |_| \_\\___/ |_| \___/ \____\___/|_____|
%
\chapter{Protocol Description}\label{chap:protocol}
{\setlength{\epigraphwidth}{.5\textwidth}
\epigraph{Protocol is etiquette with a government expense account.}{---Judith Martin}}
\LJversal{T}{he protocol} used to control the busylight device is very simple.
Commands are expressed largely in plain \ascii\ characters and are executed immediately
as they are received.\footnote{Technically, they may even be executed \emph{while} they
are being received.}
%It is not necessarily required for a command to be fully received first,
%so it is possible that the unit will have started operating on part of the command (e.g.,
%moving the current cursor column position) even if the rest of the command could not be
%performed.

\section{USB vs. RS-485}
The protocol used to send commands to the busylight is different depending on
whether the host is sending directly to a single device over a \acronym{USB}
cable, or to (possibly) multiple devices over an RS-485 bus network.

This was designed for the Adruino Nano controller, which only has a single
\acronym{USART}, so the busylight will differentiate between \acronym{USB}
and RS-485 commands from their respective formats. On other units with multiple
serial ports, one is dedidated to the RS-485 network.

\subsection{USB}
A busylight connected via \acronym{USB} accepts the commands just as documented
below, with the addition that each such command is terminated by a 
\z{\textasciicircum D} byte (hex value \z{04}$_{16}$).
\begin{center}
\begin{bytefield}[endianness=little,bitwidth=0.11111\textwidth]{9}
	\bitbox[]{1}{\scriptsize0}&
	\bitbox[]{1}{\scriptsize\dots}&
	\bitbox[]{1}{\scriptsize$n-1$}&
	\bitbox[]{1}{\scriptsize$n$}\\
	\bitbox{3}{\Var*{command}}&
	\bitbox{1}{\z{\textasciicircum D}}
\end{bytefield}
\end{center}

If there is an error parsing or executing a command, the busylight will ignore
all subsequent input until a \z{\textasciicircum D} is received, whereupon it will
expect to see the start of another command. Thus, \z{\textasciicircum D} may not
appear in any transmitted data except to terminate commands.

\subsection{RS-485}
Commands sent over RS-485 are intended to target one or more of a set of
connected busylight units over a network which may also contain other Lumos-protocol-compatible
devices, so they use a protocol compatible with that use case.

Each command begins with one of the following binary headers, depending on the
set of target busylight units which should obey the command.

\subsubsection{Single Target or All Devices}
To send a command to a single unit, begin with a single byte encoded as:
\begin{center}
	\begin{bytefield}[endianness=big]{8}
		\bitheader{0-7} \\
		\bitbox{1}{\z{1}}&
		\bitbox[tbl]{1}{\z0}&
		\bitbox[tb]{1}{\z0}&
		\bitbox[tbr]{1}{\z1}&
		\bitbox{4}{\Var*{ad}}
	\end{bytefield}
\end{center}
where \Var*{ad} is the unit's address on the bus, which must be a value in the
range 0--15. This byte is followed by any command as described below. 
If the global address \Var{ad$_G$} is given as the \Var*{ad} value, then all busylight units
which have that set as their global address will obey the command.

\subsubsection{Multiple Targets}
Alternatively, a command may be targetted to multiple units by starting the command
with a multi-byte code:
\begin{center}
	\begin{bytefield}[endianness=big]{8}
		\bitheader{0-7} \\
		\bitbox{1}{\z{1}}&
		\bitbox[tbl]{1}{\z0}&
		\bitbox[tb]{1}{\z1}&
		\bitbox[tbr]{1}{\z1}&
		\bitbox{4}{\Var*{ad$_G$}}\\
		\bitbox{1}{\z0}&
%		\bitbox{3}{\color{reservedslot}\rule{\width}{\height}}&
%		\bitbox{4}{\Var*{n}}\\
		\bitbox{1}{\z0}&
		\bitbox{6}{\Var*{n}}\\
		\bitbox{1}{\z0}&
%		\bitbox{3}{\color{reservedslot}\rule{\width}{\height}}&
%		\bitbox{4}{\Var*{ad$_0$}}\\
		\bitbox{1}{\z0}&
		\bitbox{6}{\Var*{ad$_0$}}\\
		\bitbox[]{8}{$\vdots$\strut}\\
		\bitbox{1}{\z0}&
%		\bitbox{3}{\color{reservedslot}\rule{\width}{\height}}&
%		\bitbox{4}{\Var*{ad$_{n-1}$}}
		\bitbox{1}{\z0}&
		\bitbox{6}{\Var*{ad$_{n-1}$}}
	\end{bytefield}
\end{center}
where \Var*{ad$_G$} is the ``global'' device address which signals busylight units
generally (see the \z= command below).
This will send to the \Var*{n} devices addressed as \Var*{ad$_0$} through \Var*{ad$_{n-1}$}.

Note that the address of a unit is constrained to four bits (values 0--15) in the regular
command header, but the address values listed here are seven bits wide (values 0--63).
This allows you to give busylight devices address above 15, but if you do this, then
this multiple-target header must be used to send commands to them, since their addresses
won't fit in the shorter single-byte header.

\subsubsection{All Off}
As a special case, the single byte
\begin{center}
	\begin{bytefield}[endianness=big]{8}
		\bitheader{0-7} \\
		\bitbox{1}{\z{1}}&
		\bitbox[tbl]{1}{\z0}&
		\bitbox[tb]{1}{\z0}&
		\bitbox[tbr]{1}{\z0}&
		\bitbox{4}{\Var*{ad}}
	\end{bytefield}
\end{center}
will cause the busylight addressed as \Var*{ad} to turn off all \led s.
If \Var*{ad} is the \Var{ad$_G$} address, then all busylight units will turn off
all \led s.

No other command bytes need to follow; this byte is sufficient to turn off
the unit(s).

\subsubsection{Subsequent Command Bytes}
All subsequent bytes which follow the above binary headers \emph{must} have their
\acronym{MSB} cleared to \z0. 

To cover cases where a value sent as part of a command must have the \acronym{MSB}
set, we use the following escape codes:
\begin{itemize}
	\item A hex byte \z{7E}$_{16}$ causes the next byte received to have its
		\acronym{MSB} set upon receipt.
	\item A hex byte \z{7F}$_{16}$ causes the next byte to be accepted without
		any further interpretation.
\end{itemize}
Thus, the byte \z{C4}$_{16}$ is sent as the two-byte sequence \z{7E~44},
while a literal \z{7E} is sent as \z{7F~7E} and a literal \z{7F} as \z{7F~7F}.

If there is an error parsing or executing a command, the busylight will ignore
all subsequent input until a byte arrives with its \acronym{MSB} set to \z1,
whereupon it will expect to see the start of another command. 

\section{Command Summary}
\begin{table}
	\begin{center}
		\begin{tabular}{cll}\toprule
			\multicolumn{1}{c}{\bfseries Command}&
			\multicolumn{1}{c}{\bfseries Description}&
			\multicolumn{1}{c}{\bfseries Notes}\\\midrule
			\z{?} & Query discrete \led\ status & [1] [2]\\
			\z{=} & Set operational parameters & [2]\\
			\z{*} & Strobe \led s in Sequence & \\
			\z{F} & Flash \led s in Sequence & \\
			\z{L} & Light one or more \led s steady & \\
			\z{Q} & Query device status & [1] [2]\\
			\z{S} & Light one \led\ steady & \\
			\z{X} & All \led s off & \\
			\bottomrule
			\multicolumn{3}{l}{\footnotesize [1] Sends response}\\
			\multicolumn{3}{l}{\footnotesize [2] \acronym{USB} only}\\
		\end{tabular}
		\caption{Summary of All Commands\label{tbl:commands}}
	\end{center}
\end{table}

%Although the rev 2 hardware supports the ability to enable the RS-485 transmitter and send data back onto the network,
%this is not currently implemented by the firmware, and the intent is to have all devices listen passively to RS-485 traffic
%at all times.

\section{\z{*}---Strobe Lights in Sequence}
\begin{center}
\begin{bytefield}[endianness=little,bitwidth=0.11111\textwidth]{7}
%	\bitheader{0-3} \\
	\bitbox[]{1}{\scriptsize0}&
	\bitbox[]{1}{\scriptsize1}&
	\bitbox[]{1}{\scriptsize2}&
	\bitbox[]{1}{\scriptsize3}&
	\bitbox[]{1}{\scriptsize\dots}&
	\bitbox[]{1}{\scriptsize$n$}&
	\bitbox[]{1}{\scriptsize$n+1$}\\
	\bitbox{1}{\z{*}} &
	\bitbox[tlb]{1}{\Var*{led$_0$}} &
	\bitbox[tb]{1}{\Var*{led$_1$}} &
	\bitbox[tb]{1}{\Var*{led$_2$}} &
	\bitbox[tb]{1}{$\cdots$} &
	\bitbox[tbr]{1}{\Var*{led$_{n-1}$}} &
	\bitbox{1}{\z\$}
\end{bytefield}
\end{center}

Each \Var*{led} value is an \ascii\ character corresponding to a discrete
\led\ as shown in Table~\ref{tbl:lightcodes}. An \Var*{led} value of ``\z{\_}'' means
there is no \led\ illuminated at that point in the sequence.

This command functions identically to the \z{F} command (see below), except that the lights
are ``strobed'' (flashed very briefly with a pause between each light in the sequence).

\section{\z{=}---Set Operational Parameters}
\begin{center}
\begin{bytefield}[endianness=little,bitwidth=0.11111\textwidth]{8}
%	\bitheader{0-3} \\
	\bitbox[]{1}{\scriptsize0}&
	\bitbox[]{1}{\scriptsize1}&
	\bitbox[]{1}{\scriptsize2}&
	\bitbox[]{1}{\scriptsize3}&
	\bitbox[]{1}{\scriptsize4}\\
	\bitbox{1}{\z{=}} &
	\bitbox{1}{\Var*{ad}}&
	\bitbox{1}{\Var*{uspd}}&
	\bitbox{1}{\Var*{rspd}}&
	\bitbox{1}{\Var*{ad$_G$}}
\end{bytefield}
\end{center}

This command sets a few operational parameters for the unit. Once set, these will be persistent across
power cycles and reboots.

If the \Var*{ad} parameter is ``\z{\_}'' then the RS-485 interface is disabled entirely. Otherwise it is a
value from 0--63 encoded as described in Table~\ref{tbl:int063}. This enables the RS-485 interface and assigns
this unit's address to \Var*{ad}. Note that if you assign an address greater than 15 you can only address
the unit via multiple-target headers or via the global address.

The baud rate for the \acronym{USB} and RS-485 interfaces is set by the \Var*{uspd} and \Var*{rspd} values
respectively. Each is encoded as per Table~\ref{tbl:baudcodes}.

The \Var*{ad$_G$} value is an address in the range 0--15 which is not assigned
to any other device on the RS-485 network. This is used to signal that all
busylight units should pay attention to the start of the command because it might
target them either as part of a list of specific busylight units or because the
command is intended for all busylight units at once. This is encoded in the 
same way as \Var*{ad}.
If you only have one busylight or do no wish to assign a global address,
just set \Var*{ad$_G$} to the same value as \Var*{ad}.

This command may only be sent over the \acronym{USB} port.

By default, an unconfigured busylight is set to 9,600 baud with the RS-485 port disabled.
\begin{table}
	\begin{center}
		\begin{tabular}{crl}\toprule
			\bfseries Code & \multicolumn{1}{c}{\bfseries Speed} \\\midrule
			\z0 & 300\\
			\z1 & 600\\
			\z2 & 1,200\\
			\z3 & 2,400\\
			\z4 & 4,800\\
			\z5 & 9,600 & (default)\\
			\z6 & 14,400\\
			\z7 & 19,200\\
			\z8 & 28,800\\
			\z9 & 31,250\\
			\z{A} & 38,400\\
			\z{B} & 57,600\\
			\z{C} & 115,200\\
		\bottomrule
		\end{tabular}
		\caption{Baud Rate Codes\label{tbl:baudcodes}}
	\end{center}
\end{table}

\section{\z?---Query Discrete \led\ Status}
\begin{center}
\begin{bytefield}[endianness=little,bitwidth=0.11111\textwidth]{1}
	\bitheader{0} \\
	\bitbox{1}{\z{?}}
\end{bytefield}
\end{center}

This command causes the unit to send a status report back to the host to indicate
what the discrete \led s are currently showing. This response has the form:

\medskip

\begin{center}\begin{bytefield}[endianness=little,bitwidth=0.11111\textwidth]{9}
	\bitheader{0-8} \\
	\bitbox{1}{\z{L}} &
	\bitbox[tbl]{1}{\Var*{led$_0$}} &
	\bitbox[tb]{1}{\Var*{led$_1$}} &
	\bitbox[tb]{1}{\Var*{led$_2$}} &
	\bitbox[tb]{1}{\Var*{led$_3$}} &
	\bitbox[tb]{1}{\Var*{led$_4$}} &
	\bitbox[tb]{1}{\Var*{led$_5$}} &
	\bitbox[tb]{1}{\strut$\cdots$} &
	\bitbox[tbr]{1}{\Var*{led$_{n-1}$}} \\
	\bitbox{1}{\z{\$}} &
	\bitbox{1}{\z{F}} &
	\bitbox{7}{flasher status (see below)} \\
	\bitbox{1}{\z{\$}} &
	\bitbox{1}{\z{S}} &
	\bitbox{7}{strober status (see below)} \\
	\bitbox{1}{\z{\$}} &
	\bitbox{1}{\z{\textbackslash n}}
\end{bytefield}
\end{center}

Each \Var*{led$_x$} value is a single character which is ``\z{\_}'' if the corresponding \led\ is
off, or the \led's color code or position number if it is on. a ``\z{?}'' is sent if the value set
for the \led\ is invalid. One such value is sent for each \led\ installed
in the unit (typically seven for busylight units), followed by a ``\z{\$}'' to mark the end of the list.

The flasher and strober status values are variable-width fields which indicate the
state of the flasher (see \z{F} command) and strober (see \z{*} command) functions.
In each case, if there is no defined sequence, the status field will be:

\medskip

\begin{center}\begin{bytefield}[endianness=little,bitwidth=0.11111\textwidth]{2}
	\bitheader{0-1} \\
	\bitbox{1}{\Var*{run}} &
	\bitbox{1}{\strut\z{\_}}
\end{bytefield}
\end{center}

\smallskip

\noindent Otherwise, the state of the flasher or strober unit is indicated by:

\medskip

\begin{center}\begin{bytefield}[endianness=little,bitwidth=0.11111\textwidth]{7}
%	\bitheader{0-6} \\
	\bitbox[]{1}{\scriptsize0}&
	\bitbox[]{1}{\scriptsize1}&
	\bitbox[]{1}{\scriptsize2}&
	\bitbox[]{1}{\scriptsize3}&
	\bitbox[]{1}{\scriptsize4}&
	\bitbox[]{1}{\scriptsize\dots}&
	\bitbox[]{1}{\scriptsize$n+3$}\\
	\bitbox{1}{\Var*{run}} &
	\bitbox{1}{\Var*{pos}} &
	\bitbox{1}{\z{@}} &
	\bitbox[tbl]{1}{\Var*{led$_0$}} &
	\bitbox[tb]{1}{\Var*{led$_1$}} &
	\bitbox[tb]{1}{$\cdots$} &
	\bitbox[tbr]{1}{\Var*{led$_{n-1}$}} 
\end{bytefield}
\end{center}

In either case, \Var*{run} is the \ascii\ character ``\z{S}'' if the unit is
stopped or ``\z{R}'' if it is currently running.  If there is a defined sequence,
\Var*{pos} indicates the 0-origin position within the sequence of the light currently
being flashed or strobed, encoded as described in Table~\ref{tbl:int063}. 
The \Var*{led$_x$} values are as allowed for the \z{F} or \z{*}
command that set the sequence. (Regardless of the actual \z{F} or \z{*} command parameters,
the report will show symbolic color codes where possible, or numeric position codes otherwise.)

The status message sent to the host is terminated by a newline character (hex byte \z{0A}),
indicated in the protocol description above as ``\z{\textbackslash n}''.
\begin{table}
	\begin{center}
		\begin{tabular}{cc|cc}\toprule
			\multicolumn{1}{c}{\bfseries Value} &
			\multicolumn{1}{c}{\bfseries Code} &
			\multicolumn{1}{c}{\bfseries Value} &
			\multicolumn{1}{c}{\bfseries Code} \\\midrule
			0--9 & \z0--\z9 & 17--42 & \z{A}--\z{Z} \\
			10 & \z: & 43 & \z[ \\
			11 & \z; & 44 & \z\textbackslash \\
			12 & \z< & 45 & \z] \\
			13 & \z= & 46 & \z\textasciicircum \\
			14 & \z> & 47 & \z{\_} \\
			15 & \z? & 48 & \z` \\
			16 & \z@ & 49--63 & \z{a}--\z{o} \\
			\bottomrule
		\end{tabular}

		{\footnotesize (Each code is the numeric value plus 48.)}
		\caption{\ascii\ Encoded Integer Values (0--63)\label{tbl:int063}}
	\end{center}
\end{table}

This command may only be sent on the \acronym{USB} port.


\section{\z{F}---Flash Lights in Sequence}
\begin{center}
\begin{bytefield}[endianness=little,bitwidth=0.11111\textwidth]{7}
%	\bitheader{0-7} \\
	\bitbox[]{1}{\scriptsize0}&
	\bitbox[]{1}{\scriptsize1}&
	\bitbox[]{1}{\scriptsize2}&
	\bitbox[]{1}{\scriptsize3}&
	\bitbox[]{1}{\dots}&
	\bitbox[]{1}{\scriptsize$n$}&
	\bitbox[]{1}{\scriptsize$n+1$}\\
	\bitbox{1}{\z{F}} &
	\bitbox[tbl]{1}{\Var*{led$_0$}} &
	\bitbox[tb]{1}{\Var*{led$_1$}} &
	\bitbox[tb]{1}{\Var*{led$_2$}} &
	\bitbox[tb]{1}{$\cdots$} &
	\bitbox[tbr]{1}{\Var*{led$_{n-1}$}} &
	\bitbox{1}{\z\$}
\end{bytefield}
\end{center}

Each \Var*{led} value is an \ascii\ character corresponding to a discrete
\led\ as shown in Table~\ref{tbl:lightcodes}. Note that the assignment of colors
to these \led s is dependent on your particular hardware being assembled that way.
As an open source project, of course, you (or whomever assembled the unit) may choose any
color scheme you like when building the board.

An \Var*{led} value of ``\z{\_}'' means there is to be no \led\ illuminated at the
corresponding position in the sequence.
\begin{table}
	\begin{center}
		\begin{tabular}{ccl}\toprule
			\multicolumn{1}{c}{\bfseries Code*}&
			\multicolumn{1}{c}{\bfseries Light}&
			\multicolumn{1}{c}{\bfseries Color}\\\midrule
			\z{B} & L$_0$ & blue \\
			\z{R} & L$_1$ & red \\
			\z{r} & L$_2$ & red \\
			\z{Y} & L$_3$ & yellow \\
			\z{G} & L$_4$ & green \\
			\z{\_}& --- & (no \led/off) \\
			\z0--\z9&L$_0$--L$_9$&\led\ installed at physical position 0--9\\
			\bottomrule
		\end{tabular}\\
		{\footnotesize *If a unit is built with different colors in these positions, the letter codes\\ for those
		\led s will match the custom color arrangement for that unit.}
		\caption{Discrete \led\ Codes and Colors\label{tbl:lightcodes}}
	\end{center}
\end{table}

Up to 64 \Var*{led} codes may be listed. The unit will cycle through the sequence, lighting each
specified \led\ briefly before moving on to the next one. The sequence is repeated
forever in a loop until an \z{L}, \z{S} or \z{X} command is received. 

If only one \Var*{led} is specified, that light will be flashed on and off.
Setting an empty
sequence (no codes at all) stops the flasher's operation.

The sequence is terminated by either a dollar-sign (``\z{\$}'') character or the
escape control character (hex byte \z{1B}), indicated in the protocol diagram above
simply as ``\z{\$}''.
			
This command may be given in upper- or lower-case (``\z{f}'' or ``\z{F}'').


\section{\z{Q}---Query Device Status}
\begin{center}
\begin{bytefield}[endianness=little,bitwidth=0.11111\textwidth]{1}
	\bitheader{0} \\
	\bitbox{1}{\z{Q}} 
\end{bytefield}
\end{center}

This command causes the unit to send a status report back to the host to indicate
the general status of the device except for the discrete \led\ display which
may be queried using the \z? command. The response has the form:

\medskip

\begin{center}
\begin{bytefield}[endianness=little,bitwidth=0.11111\textwidth]{9}
%	\bitheader{0-8} \\
	\bitbox[]{1}{\scriptsize0}&
	\bitbox[]{1}{\scriptsize1}&
	\bitbox[]{1}{\scriptsize2}&
	\bitbox[]{1}{\scriptsize3}&
	\bitbox[]{1}{\scriptsize4}&
	\bitbox[]{1}{\scriptsize5}&
	\bitbox[]{1}{\scriptsize6}&
	\bitbox[]{1}{\scriptsize7}&
	\bitbox[]{1}{\scriptsize8}\\
	\bitbox{1}{\z{Q}} &
	\bitbox{1}{\Var*{model}} &
	\bitbox{1}{\z{=}} &
	\bitbox{1}{\Var*{ad}}&
	\bitbox{1}{\Var*{uspd}}&
	\bitbox{1}{\Var*{rspd}}&
	\bitbox{1}{\Var*{ad$_G$}}&
	\bitbox{1}{\z{\$}} &
	\bitbox{1}{\z{V}} \\
	\bitbox{2}{\Var*{hwversion}}
	\bitbox{1}{\z\$}&
	\bitbox{1}{\z{R}} &
	\bitbox{2}{\Var*{romversion}}&
	\bitbox{1}{\z{\$}}& 
	\bitbox{1}{\z{S}}& 
	\bitbox{1}{\Var*{serial}} \\
	\bitbox{1}{\z{\$}} &
	\bitbox{1}{\z{\textbackslash n}}
\end{bytefield}
\end{center}

The \Var*{model} field is ``\z{B}'' for the busylight hardware. Other codes
designate other compatible devices.

\Var*{hwversion} and \Var*{romversion} indicate the versions, respectively, of the 
hardware the firmware was compiled to drive, and of the firmware itself. Each of
these fields are variable-width and conform to the semantic version standard 2.0.0.\footnote{See
\href{https://semver.org}{semver.org}.} Each is terminated by a dollar-sign (\z\$) character (and
thus those strings may not contain dollar signs).

The \Var*{serial} field is a variable-width alphanumeric string which was set when the firm\-ware was
compiled. It should be a unique serial number for the device (although that depends on
some effort on the part of the person compiling the firmware to insert that serial number
each time). Serial numbers B000--B299 are reserved for the original author's use. This string is also
terminated with a dollar sign.

The \Var*{ad}, \Var*{uspd}, and \Var*{rspd} values are as last set by the \z= command (or the factory
defaults if they were never changed). If the serial device is disabled, ``\z{\_}'' is sent instead of
the baud rate code. Likewise, ``\z{\_}'' is sent for the address of a device for which no address
is set. ``\z{*}'' is sent for an address that is somehow invalid.

The status message sent to the host is terminated by a newline character (hex byte \z{0A}),
indicated in the protocol description above as ``\z{\textbackslash n}''.

\section{\z{S}---Light Single \led}
\begin{center}
\begin{bytefield}[endianness=little,bitwidth=0.11111\textwidth]{2}
	\bitheader{0-1} \\
	\bitbox{1}{\z{S}} &
	\bitbox{1}{\Var*{led}} 
\end{bytefield}
\end{center}

Stops the flasher (cancelling any previous \z{F} command) and turns off all discrete
\led s. The single \led\ indicated by \Var*{led} is turned on. See Table~\ref{tbl:lightcodes}. Note that if a strobe sequence is running (via a previous \z{*} command),
it remains running.

This command may be given in upper- or lower-case (``\z{s}'' or ``\z{S}'').

\section{\z{X}---Turn off Discrete \led s}
\begin{center}
\begin{bytefield}[endianness=little,bitwidth=0.11111\textwidth]{1}
	\bitheader{0} \\
	\bitbox{1}{\z{X}} 
\end{bytefield}
\end{center}

Turns off the flasher, strober, and all discrete \led s.

This command may be given in upper- or lower-case (``\z{x}'' or ``\z{X}'').

\input pinouts

\indexintoc

\printindex
\end{document}
